\documentclass{article}
\usepackage[utf8]{inputenc}

\usepackage{hyperref}
\hypersetup{
    colorlinks=true,
    linkcolor=black,
    filecolor=magenta,      
    urlcolor=cyan,
}
 
\urlstyle{same}

\pagenumbering 

\title{VR Supermarket}
\author{Vinamra Rai }
\date{June 2017}

\begin{document}

\maketitle

\tableofcontents
\pagebreak


\section{About the Project}

This is a Summer Project under the Programming Club of IIT Kanpur.

In today's world, where we are too busy with tonnes of stuff, won't it be interesting if we could buy items of our daily needs while sitting at our home and still have the same experience of choosing our items like we do in supermarket? So \textbf{VR Supermarket} is the answer to this question. We aim to create an Android App having 3D virtual experience for customers so that they can visit their nearest grocery or general store while sitting on their couch.

\section{Week by Week Preparation Record}
\subsection{Week 0 (Before 20 May)}
Learned Python and Terminal commands using 'Learn Python the Hard Way' by Zed Shaw and completing it till Exercise 42. \\
\url{https://github.com/rvinamra/SummerProjectPrep}

\subsection{Week 1 (21 May - 27 May)}
Learned about about how does a VR device works by following a small course on Virtual Reality and its Principles on Udacity and got introduced to Unity 3D software.
Followed Jimmy Vegas' tutorials on YouTube to create a simple game environment to get a hands on experience on creating a 3D environment.

\subsection{Week 2 (May 28 - 3 June)}
Continued with learning more about Unity 3D software.
Got to know about Firebase, and tried learning more about it. 

\subsection{Week 3 (4 June - 10 June)}
Continued with creation of couple of sample projects on Unity 3D.

\subsection{Week 4 (11 June - 17 June)}
Started creating a prototype of Departmental store to test various functions on a small scale before starting with full scale development.


\end{document}
